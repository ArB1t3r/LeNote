\documentclass{leptc}
\begin{document}


\chap{双语彩色笔记模版}

作者:\href{mailto:alileptc@gmail.com}{LePtC}

项目主页:\url{https://github.com/LePtC }

Last compiled on {\yyyymmdddate\today} at \currenttime


\chap{安装}

\ent[install TeX]{安装\TeX 系统} 直接安装 \href{http://www.ctex.org/CTeXDownload }{CTeX Full} 或 \href{http://www.ctan.org/tex-archive/systems/texlive/Images/ }{TeXLive iso} ,前者是把 \file{leptc.cls} 放到 \code{CTeX/MiKTeX/tex/latex/} 目录下,然后在 MiKTeX 的 Settings 里面点 Refresh FNDB 即可,后者是在 \code{texlive/2014/texmf.cnf} 末尾加上 \code{TEXMFLOCAL = $SELFAUTOPARENT/../texmf-local,E:/anypath}, 然后把\file{leptc.cls} 放到 \\ \code{anypath/tex/latex/misc} 中,在命令行执行 \code{texhash} 即可

\ent[compiler]{编译器} 
只有 latex+dvipdfmx 或 xelatex 编译出的 pdf 能正确复制,前者请参考文件 \file{leptc.sty}

dvipdfmx 方案本狸已停止更新,推荐使用 \XeTeX 方案, xelatex 的编译命令及常用选项:

\code{xelatex --quiet --synctex=1 -interaction=nonstopmode $(NAME_PART).tex}

\com{xelatex.exe 等编译器均在 \code{CTeX/MiKTeX/miktex/bin/} 或 \code{texlive/2014/bin/win32} 目录下,如果命令行没有此命令,可在命令中输入 exe 的完整路径,或手动将路径添加到系统的环境变量并重启}

\ent[editor]{编辑器} \href{http://tex.stackexchange.com/questions/339/latex-editors-ides }{各种编辑器的比较},熟悉哪个就用哪个好啦,初学者可以就用安装\TeX 系统时带的 TeXworks 

阿狸用的是 notepad++, synctex 需要借助一个 dde 插件 cl-2-dde-1.0.exe ,其它编辑器各有不同的设置方法

\ent[reader]{阅读器} 推荐使用 \href{http://blog.kowalczyk.info/software/sumatrapdf/download-free-pdf-viewer-cn.html }{SumatraPDF} 来查看 pdf,有 \href{http://xhmikosr.1f0.de/sumatrapdf/ }{64 位版本}\com{非官方的}

支持 synctex,需在 \code{InverseSearchCmdLine} 里填入相应编辑器反向查找的命令

\ent[tex file]{tex 文档}新建filename.tex,存为 UTF-8 无 BOM 格式,开头为 \verb|\documentclass{leptc}|,然后就可以 \\ \verb|\begin{document}| 闭着眼睛写啦,喵 \tld

\com{待解决:文档名不能有空格否则不能识别,不能有中文否则会报错}




\chap{章节}

\begin{tabular}{lcll}

	章节
	&\com{效果见右上方\eq{\nearrow} }
	&\verb|\chap{中文}|
	&\com{说明\eq{\downarrow} }\\
	
	 
	&\ent[entry left]{词条} \hspace{30pt} \entc[entry center]{词条} \hspace{20pt}	
	&\verb|\ent[entry]{词条} | 
	&居中用 \verb|\entc| 		\\

	 
	&\eng[English translation]{注英文} 	
	&\verb|\eng[English]{正文} | 
	& \verb|\engs| \engs[translation]{标在右侧} 		\\

	inline公式 
	&\eq{f(x,y)=\frac{\e^x}{y}}
	&\verb|\eq{\frac{\e^x}{y}}|
	&放弃用\verb|$$|,配对容易出错	\\

	display公式 
	&\eqd{f(x,y)=\frac{\e^x}{y}}
	&\verb|\eqd{\frac{\e^x}{y}}|
	&修改公式模式只需加一个 \verb|d|即可	\\

	 
	&\com{注释}
	&\verb|\com{注释}|
	&仿铅笔的颜色	\\

	证明
	&\eq{\vec{v}=\prv{\od{}{t}(r \ve{r})=}\dot r\ve{r}+r\dot \theta\ve{\theta}\quad}
	&\verb|\prv{blabla=}|
	&灰色的优先级低于自动高亮 	\\

	笔记间的链接 
	&\link{颜色}
	&\verb|\link[笔记名]{章节名}|
	&同一笔记内的链接笔记名可省略	\\

% \fig[0.5]{picture.png}

\end{tabular}




\chap{颜色}

模版对以下情况做自动高亮:

\ \\
\begin{tabular}{lccl}

	函数名橙色
	&\eq{\sin(x+y),\exp[x+y]}
	&\verb|\e^{x+y},\exp[x+y]|
	&自然对数 \eq{\e^x} 也变橙色,命令为 \verb|\e| \\
	
	算符绿色
	&\eq{\dif x,\Dif x,\delta x,\Delta x,\nabla x}
	&\verb|\dif x,\delta x,\nabla x|
	&默认自动高亮,不高亮用 \verb|\olddelta| \\
	
	物理单位紫色
	&\eq{6.67\E{-11}\uni{m^3/(kg\cdot s^2)}}
	&\verb|\uni{m^3/(kg\cdot s^2)}|
	&虚数单位 \eq{\ii} 也变紫色,命令为 \verb|\ii| \\
	
\end{tabular}

\ \\
但字母作大型运算符\com{如\eqd{\min_{i=1}^n}}不做高亮,不易混淆的符号型算符\com{如\eqd{\sqrt{\ }}}不做高亮


\chap{字体}

正文默认字体: Adobe 仿宋,\textbf{词条 Adobe 黑体},英文 Times New Roman,\engs[Verdana]{英文翻译}

打字机 \verb|\texttt{}|用于源代码: \file{file.tex}

\ \\
为了避免命名空间冲突,为了世界的和平,强迫症如下规定数学字体的含义:

\ \\
\begin{tabular}{lcl}

	所有变量、粒子符号为斜体
	&\eq{x,y,z,r,v,a,e,n,p}
	&\com{公式环境下默认为斜体} \\

	其它字母、元素符号为正体
	&\eq{\kb,\NA,F\inter,\cc,\mathrm{He}}
	&\verb|\mathrm{}| \\

	粗体注册为群论
	&\eq{\mathbf{U}(n),\mathbf{SU}(2),\mathbf{T}^\alpha}
	&\verb|\mathbf{}| \\
	
	双线体注册为数域
	&\eq{\mathbb{N,Z,Q,A,R,C,H}}
	&\verb|\mathbb{}| \\
	
	花体注册为泛函
	&\eq{\mathcal{L,F}}
	&\verb|\mathcal{}| \\
	
\end{tabular}


\chap{数学}

\begin{tabular}{lcl}

	矢量
	&\eq{\vec{OA},\vec{p_c}',\vecd{p},\ve{r}}
	&\verb|\vec{OA},\vec{p_c}',\vecd{p},\ve{r}|\\

	双矢量
	&\eq{\dvec{r},\dvecd{P}}
	&\verb|\dvec{r},\dvecd{P}|\\
	
	导数,偏导数
	&\eqd{\od{y}{x},\pd[2]{L}{x},\md{L}{4}{x}{2}{y}{2}}
	&\verb|\od{y}{x},\pd[2]{L}{x},\md{L}{4}{x}{2}{y}{2}|\\

	某处的导数
	&\eq{\odat{y}{x}{x_0},}
	\eqd{\odat{y}{x}{x_0},\pdat{L}{x}{y,z}}
	&\verb|\odat{y}{x}{x_0},\pdat{L}{x}{y,z}|\\

	矢量微分
	&\eq{\nabla x,\nablad \vec x,\nablat \vec x,\nablas x}
	&\verb|\nabla x,\nablad \vec x,\nablat \vec x,\nablas x|\\

\end{tabular}

\ \\
太多了 ... 慢慢写


\chap{其它}

\url{http://tex.stackexchange.com/ }

\href{http://linux-wiki.cn/wiki/zh-hans/LaTeX%E4%B8%AD%E6%96%87%E6%8E%92%E7%89%88%EF%BC%88%E4%BD%BF%E7%94%A8XeTeX%EF%BC%89 }{ LaTeX中文排版(使用XeTeX)}

\href{http://www.wikibooks.org }{维基}






\end{document}