\documentclass{leptc}
\begin{document}


\chap{双语彩色笔记模版}

作者:\href{mailto:alileptc@gmail.com}{LePtC}

项目主页:\url{https://github.com/LePtC/LeNote }

笔记主页:\url{http://leptc.github.io/lenote }

使用 \href{http://opensource.org/licenses/MIT}{MIT 开源协议}

Last compiled on {\yyyymmdddate\today} at {\hhmmsstime} [UTC+8]


\chap{安装}

\ent[install TeX]{安装\TeX 系统}
Windows 系统可选择安装
\href{http://miktex.org/download}{MiKTeX}
然后选择自动安装缺失的包,或直接安装
\href{http://www.ctex.org/CTeXDownload }{CTeX Full}
或 \href{http://www.ctan.org/tex-archive/systems/texlive/Images/ }{TeXLive iso} ,
前两者是把 \code{leptc.cls} 放到
\code{CTeX/MiKTeX/tex/latex/} 目录下,
然后在 MiKTeX 的 Settings 里面点 Refresh FNDB 即可,
后者是在 \code{texlive/2014/texmf.cnf} 末尾加上
\\ \code{TEXMFLOCAL = $SELFAUTOPARENT/../texmf-local,E:/blabla/(anypath)},
\\然后把\code{leptc.cls} 放到
\code{(anypath)/tex/latex/misc} 这个路径中,
在命令行执行 \code{texhash} 即可

\ent[compiler]{编译器}
只有 latex+dvipdfmx 或 xelatex 编译出的 pdf 能正确复制,
前者请参考文件 \code{leptc.sty}

dvipdfmx 方案本狸已停止更新,推荐使用 \XeTeX 方案,
xelatex 的编译命令及常用选项:

\code{xelatex --quiet --synctex=1 -interaction=nonstopmode $(NAME_PART).tex}

xelatex 需要多编译几遍才能正确生成书签,
详见 \href{https://github.com/LePtC/LeNote}{项目主页}的
\code{compile} 文件夹

\com{xelatex.exe 等编译器均在
\code{CTeX/MiKTeX/miktex/bin/}
或 \code{texlive/2014/bin/win32} 目录下,
如果命令行没有此命令,可在命令中输入 exe 的完整路径,
或手动将路径添加到系统的环境变量并重启}

\ent[editor]{编辑器}
\href{http://tex.stackexchange.com/questions/339/latex-editors-ides }{各种编辑器的比较},
熟悉哪个就用哪个好啦,
初学者可以就用安装\TeX 系统时带的 TeXworks

有关编辑器不同的设置方法见\href{https://github.com/LePtC/LeNote}{项目主页}的 \code{README.md}

\ent[reader]{阅读器} 推荐使用
\href{http://blog.kowalczyk.info/software/sumatrapdf/download-free-pdf-viewer-cn.html }{SumatraPDF}
来查看 pdf, 有
\href{http://xhmikosr.1f0.de/sumatrapdf/ }{64 位版本}
\com{非官方}

支持 synctex,需在 \code{InverseSearchCmdLine} 里填入相应编辑器的反向查找命令

Notepad++:
\code{\"C:\\Program\ Files\ (x86)\\Notepad++\\notepad++.exe\" -n\%l \"\%f\"}

Sublime:
\code{\"C:\\Program\ Files\\Sublime\\sublime_text.exe\" \"\%f:\%l\"}

\ent[tex file]{\TeX 文档}
新建filename.tex,存为 UTF-8 无 BOM 格式,
开头为 \verb|\documentclass{leptc}|,
然后就可以在 \verb|\begin{document} ... \end{document}| 之间写正文啦,
喵 \tld

\com{待解决:文档名不能有空格否则不能识别,
不能有中文否则会报错}




\chap{章节}

\begin{tabular}{lcll}

	章节
	&\com{效果见右上方\eq{\nearrow} }
	&\verb|\chap{中文}|
	&\com{说明\eq{\downarrow} }\\

  双语词条
	&\ent[\B Superconducting \B{QU}antum \B Interference \B Device]{超导量子干涉器}
	&\verb|\ent[\B Entry]{词条} |
	&居中用 \verb|\entc| 		\\

  双语正文
	&\eng[English translation]{注英文}
	&\verb|\eng[English]{正文} |
	& 用 \verb|\engr| 则英文标在右侧 		\\

  标签
	&\enl{标签}
	&\verb|\enl{标签} |
	& 用于\enl{例},\enl{定理},\enl{推论}等		\\

	inline公式
	&\eq{f(x,y)=\frac{\e^x}{y}}
	&\verb|\eq{\frac{\e^x}{y}}|
	&长公式不用 \verb|$$|, 括号便于配对	\\

	display公式
	&\eqd{f(x,y)=\frac{\e^x}{y}}
	&\verb|\eqd{\frac{\e^x}{y}}|
	&修改公式模式只需加一个 \verb|d|即可	\\

  圆括号表注释
	&\com{注释}
	&\verb|\com{注释}|
	&多行注释: \verb|\coms{注\\释}|	\\

	方括号表证明
	&\eq{\vec{v}=\prv{\od{}{t}(r\ve{r})=}\dot{r}\ve{r}+r\dot\theta\ve{\theta}}
	&\verb|\prv{blabla=}|
	&灰色的优先级低于自动高亮 	\\

	尖括号表链接
	&\link{颜色}
	&\verb|\link[笔记名]{章节名}|
	&同一笔记内的链接笔记名可省略	\\

	贴图
	&\figin[0.05]{ali}
	&\verb|\fig[相对宽度]{图片名}|
	&内置: \verb|\figin| 多图并排: \verb|\figgg|	\\

\end{tabular}


\chap{实例}


\com{本笔记均指实数域} \ent[orthogonal group]{正交群}
\eq{\mathbf{O}(n)}
需 \eq{\frac12n(n-1)} 个独立参数
\prv{约束方程\eq{O^TO=I}上下三角的$=0$对称}


\eq{\mathbf{O}(n)=\mathbf{SO}(n)\otimes\{I,-I\}}
\prveq{\abs{O}=\pm1}
\enl{例}
\eq{\mathbf{O}(1)=\{\pm1\},\ \mathbf{SO}(1)=\{1\}}

\ent{二维空间转动群}
\eq{\mathbf{SO}(2)=\{R_z(\theta)|-\pi\le\theta\le\pi\}}
\enl{例}
\eq{\mathbf{D}_n} 是 \eq{\mathbf{O}(2)} 的离散子群
\com{反射对应行列式 $-1$}

\com{参数群可用数学分析方法}
\prvs{
由于\eq{\mathbf{SO}(2)}阿贝尔,表示一维,设 \eq{A=\{a(\theta)\}},
已知乘法关系为 \eq{a(\theta_1+\theta_2)=a(\theta_1)a(\theta_2)},
两边对\eq{\theta_1}求导后令\eq{\theta_1=0},
得\eq{a'(\theta_2)=a(\theta_2)a'(0)},
为使幺正取\eq{a'(0)=\ii m}纯虚,解得\eq{a(\theta)=\e^{\ii m\theta}},
由周期性\eq{a(\theta)=a(\theta+2\pi)}\com{费米子是\eq{+4\pi}},
得\eq{m\in\mathbb{Z}},然后证完备
}

\ent[three dimensional rotation group]{三维空间转动群}
\eq{\mathbf{SO}(3)\nors\mathbf{O}(3)},
均由3个\ent{群参数}表示 \com{独立,实数}, 群元素写法:

\N1 \eq{R_{(\theta,\varphi)}(\psi),\ 0\le\psi\le\pi}
\to 映射到半径 \eq{\pi} 球面上 \eq{(\psi,\theta,\varphi)}
\com{球面上的点二对一 \eq{R_n(\pi)=R_{-n}(\pi)}} \link{拓扑}





\chap{图片混排}

图片混排的命令为 \verb|\figr{ali.jpg}{0.1}{很多行文字}|, 实例 \eq{\downarrow}
\ \\

\figr{natural.png}{0.22}
{
\ent[arc length]{弧长} \eq{s=s(t),\ \vec{r}=\vec{r}(s)}
\com{可任意选定 \eq{s} 的零点和正向,与运动方向无关}

\ent[tangential]{切向} \eq{\ve{t}=\frac{\dif \vec{r}}{\dif s}},
\eq{\od{}{ \theta}\ve{t}=\ve{n}\ \to}
\ent[normal]{法向}指向曲线凹侧, \eq{\od{}{\theta}\ve{n}=-\ve{t}},
\eq{\ved{t}=\od{\ve{t}}{\theta}\od{\theta}{s}\dot s=\ve{n}\frac{1}{\rho}v}

\eq{\vec{v}=\dot s\ve{t}},
\eq{\vec{a}=\ddot s\ve{t}+\frac{v^2}{\rho}\ve{n}},
\ent[curvature radius]{曲率半径} \eq{\rho=\od{s}{\theta}=(1+y'^2)^{\frac{3}{2}} / \abs{y''}},
常用 \eq{a_t=\dot v=\od{v}{s}v}

加速度既反映速度大小也反映方向变化
\eq{a_t=\od{v}{t},\ a_n=\frac{v^2}{\rho},\
a=\sqrt{a_t^2+a_n^2},\ \tan\theta=\frac{a_n}{a_t}}
}


\chap{表格混排}

表格混排的命令为 \verb|\tabr[0.4]{很多行表格}{很多行文字}|, 实例 \eq{\downarrow}
\ \\


\tabr[0.72]{
\enl{性质} 同类元素的特征标相等 \com{记类中元素个数为 \eq{n_i}, 求和公式中可合并}

群的$\forall\ne$IUR的个数等于群中类的个数 \eq{r} \to 特征标表是方阵

\ent{第一正交性关系} 特征标表各行正交
\eq{\frac{1}{n}\sum^r n_i \chi^{(p)*}(g) \chi^{(q)}(g)=\delfun_{pq}}

\ent{第二正交性关系} 特征标表各列正交
\eq{\frac{n_i}{n}\sum^r_p \chi^{(p)*}(g_i) \chi^{(p)}(g_{i\co})=\delfun_{ii\co}}

}{

\begin{tabular}{|c|c|c|c|}
\hline
  特征标 &\eq{e} &\eq{r_1,r_2} &\eq{a,b,c} \\
\hline
  \eq{\chi^S} &1 &1 &1 \\
  \eq{\chi^A} &1 &1 &\eq{-1} \\
  \eq{\chi^\Gamma} &2 &\eq{-1} &0 \\
\hline
\end{tabular}
}


\chap{颜色}

模版对以下情况做自动高亮:
\prv{更新:绿色为注释专用, 算符改用橙色, 章节由红色改为紫色}
\ \\



\hspace{-10pt}
\begin{tabular}{lccl}

  推导为绿色
  &\eq{\to \ns \Rightarrow}
  &\verb|\to \ns \Rightarrow|
  & \\

	函数名橙色
	&\eq{\sin(x+y),\exp[x+y]}
	&\verb|\e^{x+y},\exp[x+y]|
	&自然对数 \eq{\e^x} 变色,命令为 \verb|\e| \\

	算符\sout{绿色}
	&\eq{\dif x,\Dif x,\delta x,\Delta x,\nabla x}
	&\verb|\dif x,\delta x,\nabla x|
	&默认高亮,不高亮用 \verb|\olddelta| \\

	物理单位蓝色
	&\eq{\oC,6.67\E{-11}\uni{m^3/(kg\cdot s^2)}}
	&\verb|\uni{m^3/(kg\cdot s^2)}|
	&虚数单位 \eq{\ii} 变色,命令为 \verb|\ii| \\

\end{tabular}



\chap{字体}

正文默认字体: Adobe 仿宋,\textbf{词条 Adobe 黑体},
英文 Times New Roman,\engr[Verdana]{英文翻译}

\prv{2015.05 更新:为改善斜杠的显示 \ent{例/例}, 黑体字体改为方正准圆}
\ \\

为了避免命名空间冲突,为了世界的和平,强迫症如下规定数学字体的含义:
\ \\

打字机体 \verb|\texttt{}|用于源代码: \code{file.tex}
\ \\

\begin{tabular}{lcl}

	所有变量、粒子符号为斜体
	&\eq{x,y,z,r,v,a,e,n,p}
	&\com{公式环境下默认为斜体} \\

	其它字母、元素符号为正体
	&\eq{\Ek,\kB,\NA,F\inter,\cc,\ce{He}}
	&\verb|\mathrm{}| \\

	双线体注册为数域
	&\eq{\mathbb{N,Z,Q,A,R,C,H}}
	&\verb|\mathbb{}| \\

	花体注册为泛函 %和大O记号?
	&\eq{\mathcal{L,F,Z}}
	&\verb|\mathcal{}| \\

  粗体注册为群
  &\eq{\mathbf{D}_n,\mathbf{U}(n),\mathbf{SO}(3)}
  &\verb|\mathbf{}| \\

  哥特体注册为代数
  &\eq{\mathfrak{su}(n),\mathfrak{so}(3)}
  &\verb|\mathfrak{}| \\

  特殊符号
  &电动势 \eq{\emf}
  &\verb|\emf| 使用 \verb|\mathscr{}| \\

\end{tabular}



\chap{其它符号范例}

\begin{tabular}{lcl}

  大圈小圈
  &\N1 \N2 \n1 \n2
  &\verb|\N1 \N2 \n1 \n2|\\

  区分求导/撇
  &\eq{y',y\co,y\co[x]}
  &\verb|y',y\co,y\co[x]|\\

	矢量
	&\eq{\vec{OA},\vec{p_c}',\vecd{p},\ve{r}}
	&\verb|\vec{OA},\vec{p_c}',\vecd{p},\ve{r}|\\

	张量
	&\eq{\vvecd{T},\vvvec{\varepsilon}}
	&\verb|\vvecd{T},\vvvec{\varepsilon}|\\

	矢量算符
	&\eq{\hatv{p},\hatvs{S}}
	&\verb|\hatv{p},\hatvs{S}|\\

  矢量微分
  &\eq{\nabla x,\nablad \vec x,\nablat \vec x,\nablas x}
  &\verb|\nabla x,\nablad \vec x,\nablat \vec x,\nablas x|\\

\vspace{3pt}\hspace{-4pt}
  导数,偏导数
  &\eqd{\od{y}{x},\pd[2]{L}{x},\md{L}{4}{x}{2}{y}{2}}
  &\verb|\od{y}{x},\pd[2]{L}{x},\md{L}{4}{x}{2}{y}{2}|\\

	某处的导数
	&\eq{\odat{y}{x}{x_0},}
	\eqd{\odat{y}{x}{x_0},\pdat{L}{x}{y,z}}
	&\verb|\odat{y}{x}{x_0},\pdat{L}{x}{y,z}|\\

\vspace{3pt}\hspace{-4pt}
	圈积分
	&\eqd{\oiint_S \vec{B} \cdot\dif \vec{S}= \oint_L \vec{A} \cdot\dif \vec{l}}
	&\verb|\oiint_S \oint_L|\\

	矩阵,行列式
	&\eq{\mat[0.8]{1&0\\0&1},\matd[0.8]{-a&b\\c&-d}}
	&\verb|\mat{1&0\\0&1},\matd{-a&b\\c&-d}|\\

  左花括号
  &\eq{\delta _{ij} = \leftB[2]{\matn{1 &(i = j)\\ 0 &(i \ne j)}}}
  &\verb|\leftB[行数]{\matn{1 &(i = j)\\ 0 &(i \ne j)}}|\\

	推导上加字
	&\eq{\xlongequal{\text{归一}}, \xrightarrow{\times a^2}}
	&\verb|\xlongequal{\text{}} \xrightarrow{}|\\

\end{tabular}

\ \\
太多了 ... 慢慢写



\chap{学习网站}

\url{http://tex.stackexchange.com/ }

\href{http://linux-wiki.cn/wiki/zh-hans/LaTeX%E4%B8%AD%E6%96%87%E6%8E%92%E7%89%88%EF%BC%88%E4%BD%BF%E7%94%A8XeTeX%EF%BC%89 }{ LaTeX中文排版(使用XeTeX)}

\href{http://www.wikibooks.org }{维基 book}






\end{document}
