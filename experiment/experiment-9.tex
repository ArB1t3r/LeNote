% !Mode:: "TeX:UTF-8"
\documentclass{leptc-exp}


\begin{document}

\begin{titlepage}

% 首行的位置往上调整。但vspace前面需要有东西才会起效。
\phantom{Start!}
\vspace{-1.7cm}
\begin{flushleft}

清华大学物理系\\[0.2cm]
高等物理实验专题\\[4.2cm]

% Title
{ \Large \bfseries 核物理类}\\[0.4cm]
{ \HUGE \bfseries 康普顿散射实验}\\[0.4cm]
{ \huge \bfseries 实验报告}

\end{flushleft}

\vfill

\begin{flushright}
{\large
% \pillar:使用一种统一的方法提高行高
\newcommand{\pillar}{ {\Huge \phantom{A}} }

\begin{tabular}{lp{5pt}c}
\pillar 姓\qquad 名 && 喵喵喵 \\\cline{3-3}
\pillar 学\qquad 号 && 20143110XX \\\cline{3-3}
\pillar 班\qquad 级 && 物研141 \\\cline{3-3}
\pillar 实验日期 && 2014年11月18日 \\\cline{3-3}
\pillar 报告日期 && \today \\\cline{3-3}
\end{tabular}
}
\end{flushright}
\end{titlepage}

\ \\
\afterpage{\blankpage}

\twocolumn[\begin{@twocolumnfalse}
\begin{cabstract}
	简要概述主要实验内容和实验结果,内容约 200-400 字左右,以文章段落格式书写,包括由 xx 基本原理研究 xx 课题;采用了 xx 手段(过程);得到了 xx 结果。
\end{cabstract}
康普顿散射;康普顿效应

\ \\
\end{@twocolumnfalse}]


% 正文

\section{引言}

包括本实验的历史发展、目前该领域相关技术的研究进展和应用,本实验的目的和意义。

引言写作温馨提示:

1 实验目的和意义不能照抄实验讲义,请用叙述性语言写自己做实验所体会到的实验目的、意义。

2 引言中关于目前与本实验有关领域的研究进展和应用,最好自己上网查阅一两篇综述文献做大概的了解,查文献并对文献总结是做科研必备的基本功,希望在近物实验中有所体验。文中引用的结论性文字要标注参考文献,须加[]一般置于右上角。\upcite{landau2011survey} \upcite{wang1995mcml} \upcite{nuclear}

3 引言中一般不写实验原理,最好把原理放在结果分析部分,用原理去分析和解释实验现象和结果。


\section{实验}

介绍用什么型号的实验仪器在什么样的实验条件下做了哪些实验内容。如果实验仪器是自己设计的或对通用仪器做特别的改造,实验结果与你的设计改造密切相关,或者实验方法比较特别,在此要详细说明你实验或方法的独特之处。如果使用常规仪器做常规测量就不必详细叙述了,只给出实验条件即可。

说明:实验内容不是指实验操作步骤,要对内容用自己的语言概括总结。


\section{实验结果及讨论}

这部分是实验报告的重点,先给出在不同的实验条件得到各种实验结果或观察到不同的实验现象,然后对结果和现象进行分析讨论。讨论部分包括实验结果的物理解释,实验数据与理论结果对比的讨论、实验误差的分析等。
这部分
写作要求 : 文字叙述简明通顺,图表、公式规范,实验结果合理。


\subsection{数据处理}

数据处理时写明所用的公式和数据处理方法,并注意有效数字的位数,实验结果尽量以图、表形式展示;图、表格式规范,大小适中。已转化为图表的数据表格可作为附录放在参考文献后面。


\subsection{插图、表}

按次序做编号,写在图的下方,并用文字说明图的名称。

\fig[0.4]{model.png}{
\caption{光子在介质中散射的随机游走模型}
\label{fig-model}
}

1 图号:报告中所有图要统一编号

2 图题:所有图必须有图题

3 图坐标:数据图的纵横轴须标明分度、物理量和单位

4 数据点:标明数据点,必要时标明误差棒


实验报告中表格的规范

1 表号:所有表格统一编号

2 表题:所有表格必须有题目

3 数据说明:标明每一栏数据的物理意义及其单位



\subsection{公式}

\eqn{x+y}

以插入公式的形式书写,以阿拉伯数字连续编号。公式中出现的
符号在第一次出现时说明其物理意义和单位。

如果有多个实验项目, 按实验项目的顺序写,每一个实验项
目要将实验设置,结果,分析放在一个大段中表述。


\subsection{往届学生报告中常见问题与不足}

1  只给数据和结果,没有对结果所反映的物理规律做理论分析、
误差分析

2  数据及结果直接给出,没有必要的实验条件说明和所用物理计
算公式

3  结果以图、表给出时,图、表不规范(缺少坐标分度、物理量
及其单位等)



\section{结论}

200-400 字,概要实验方法,采用手段,得到的结论。

文章整体篇幅不超过 8 页


% 参考文献
\section{参考文献}

\bibliography{experiment-9}


% 附录
\appendix
\twocolumn[\begin{@twocolumnfalse}
\section{原始数据}

1 原始数据变成电子文本附在报告正文后面

2 对本实验的总结,对实验中发现的问题提出自己的建议。

\end{@twocolumnfalse}]

\end{document}
